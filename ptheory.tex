\documentclass{jsarticle}
\usepackage[dvipdfmx]{graphicx}
\usepackage[dvipdfmx]{color}
\usepackage{amsmath,amssymb,fancybox,ascmac,booktabs,diagbox,array,otf}
\usepackage{minted}
\setminted{
  linenos,
  breaklines,
  frame=lines,
  framesep=2mm,
}


\author{}
\title{確率論レポート課題}
\date{提出日:\today}
\begin{document}
\maketitle
%概収束,確率収束,p 次平均収束,分布収束について,定義をまとめよ.
\section{}
\begin{itembox}[l]{問題}
   概収束,確率収束,p次平均収束,分布収束について,定義をまとめよ.
\end{itembox} 

\subsection*{概収束}
(実数値)確率変数列$(X_n)_{n\in \mathbb{N} }$と確率変数列$X$に対して
\begin{equation*}
    P(\lim_{n \rightarrow \infty}X_n = X)=1
\end{equation*}
となるとき, $X_n$は$X$に「概収束(Almost sure Convergence)」するといい, $X_n \xrightarrow{a.s} X (n\rightarrow \infty)$とかく.

\subsection*{確率収束}
$(X_n)_{n\in \mathbb{N} }$が任意の$\epsilon > 0$に対して,
\begin{equation*}
    \lim_{n \rightarrow \infty} P (|X_n-X|>\epsilon)
\end{equation*}
となるとき, $X_n$は$X$に「確率収束(Convergence in Probability)」するといい, 
\begin{gather*}
    P \lim_{n \rightarrow \infty}X_n = X\\
    X_n \xrightarrow{P} X (n \rightarrow \infty)
\end{gather*}
とかく. 
\subsection*{p次平均収束}
$(X_n)_{n\in \mathbb{N} }$, $X$は各$n \in \mathbb{N} $に対してある$p>0$で$E[|X_n|^p]=0$とする. このとき, $$\lim_{n\rightarrow0}E[|X_n-X|^p]=0$$ならば, $X_n$は$X$に「p次平均収束(Convergence in $L^p$)」あるいは$L^p$収束するといい,$$X_n\xrightarrow{L^p}X(n\rightarrow\infty)$$と表す.

\subsection*{分布収束}
$(X_n)_{n\in \mathbb{N} }$, $X$が$\mathbb{R} $上の任意の有界連続関数$f$に対して
$$\lim_{n\rightarrow\infty}E[|f(X_n)|]=E[f(X)]$$
を満たすとき, $X_n$は$X$に「分布収束(Convergence in Distribution)」または「法則収束(Convergence in Law)」するといい,
\begin{gather*} 
X_n \xrightarrow{d}X(n\rightarrow\infty)\\
X_n \Rightarrow X (n\rightarrow \infty)
\end{gather*}
などと表す.

%概収束,確率収束,p 次平均収束,分布収束について,定義をまとめよ.
%上記で述べた,各種収束の関係性をその反例や証明を含めてまとめよ.必要に応じて図などを使用してもよい.
\section{}
\begin{itembox}[l]{問題}
   上記で述べた,各種収束の関係性をその反例や証明を含めてまとめよ.必要に応じて図などを使用してもよい.
\end{itembox}
以下の表に,各種収束の関係性をまとめる.表中の記号は以下の通りである.
\begin{itemize}
    \item ○: $A\Rightarrow B$が収束することが成り立つ. (証明付き)
    \item ×: $A\Rightarrow B$が収束しないことが成り立つ. (反例付き)
    \item △: $A\Rightarrow B$が特殊な条件下で収束することが成り立つ. (条件付き証明付き)
\end{itemize}
表中の数字はそれぞれ,前者から後者への収束に関する証明や反例の番号を示す.
\begin{center}
\renewcommand{\arraystretch}{1.25} % 行の高さを少し広げる
\begin{tabular}{|c||c|c|c|c|}
    \hline
    % diagbox{左上の見出し}{右下の見出し}
    \diagbox{B}{A\phantom{0}} & $a.s$ & $p$ & $L^p$ & $d$ \\
    \hline \hline
    $a.s$ & \diagbox{\phantom{0}}{\phantom{00}} & △ &  &  \\
    \hline
    $p$ & ○ & \diagbox{\phantom{0}}{\phantom{00}} & ○ &  \\
    \hline
    $L^p$ &  &  & \diagbox{\phantom{0}}{\phantom{00}} &  \\
    \hline
    $d$ & ○ & ○ & ○ & \diagbox{\phantom{0}}{\phantom{00}} \\
    \hline
\end{tabular}
\end{center}

\subsection*{1. $X_n\xrightarrow{a.s}X \Rightarrow X_n\xrightarrow{p}X$}
仮定$X_n\xrightarrow{a.s}X$より,任意の$\epsilon >0$に対して,
\begin{equation*}
    \lim_{n\rightarrow\infty}P\left(\bigcup_{m\geq n} {m\in\Omega ||X_m{\omega}-X{\omega}|\geq 0}\right) = 0
\end{equation*}
が成り立つ.また任意の$n\in \mathbb{N}$に対して,
\begin{equation*}
    {m\in\Omega ||X_m{\omega}-X{\omega}|\geq 0}\subset \bigcup _{m\geq n} {m\in\Omega ||X_m{\omega}-X{\omega}|\geq 0}
\end{equation*}
が成り立つので単調性より,
\begin{align*}
    &\lim_{n\rightarrow\infty}P({|X_n-X|\geq \epsilon}) \\
    &\leq P\left(\bigcup_{m\geq n} {\omega\in\Omega ||X_m(\omega)-X(\omega)|\geq 0}\right) \\
    &\rightarrow 0 (n\rightarrow \infty)
\end{align*}
が成り立つ.従って$X_n\xrightarrow{p}X$が成り立つ.

\subsection*{2. $X_n\xrightarrow{L^p}X \Rightarrow X_n\xrightarrow{p}X$}
任意の$\epsilon >0$に対して,マルコフの不等式及び仮定$X_n\xrightarrow{L^p}X$より,
\begin{align*}
    P(|X_n-X|\geq \epsilon) &=P(|X_n-X|^p\geq \epsilon^p)\\&\leq \frac{E[|X_n-X|^p]}{\epsilon ^p} \\
    &\rightarrow 0 (n\rightarrow \infty)
\end{align*}
が成り立つ.従って$X_n\xrightarrow{p}X$が成り立つ.
\subsection*{3. $X_n\xrightarrow{p}X \Rightarrow X_n\xrightarrow{d}X$}
$X_n\xrightarrow{p}$より,任意の$\epsilon >0$に対して,
\begin{align*}
    &\phantom{=}\forall \epsilon >0, \lim_{n\rightarrow\infty}P(|X_n-X|\geq \epsilon)=1 
    \\&\Leftrightarrow \forall \epsilon >0,\forall \delta >0, \exists N\in \mathbb{N},\\&\phantom{=====} \forall n\geq \mathbb{N}, [n\leq N \Rightarrow1-P(|X_n-X|\geq\epsilon)<\delta]
\end{align*}
であるから,ある$N\in\mathbb{N}$が存在して,$n\geq N$ならば$P(|X_n-X|\leq\epsilon)>1-\delta$を満たす.したがって任意に$F$の連続点$x\in\mathbb{R}$をとると,補題1より$Y$を$X_n$にすると,
\begin{equation*}
    |F_n(x)-F(x)|\leq |F(x+\epsilon)-F(x-\epsilon)|+\delta\;\;\;(n\geq N)
\end{equation*}
が成り立ち,xは$F$の連続点であるから,$$\lim_{\epsilon\downarrow 0}|F(x+\epsilon)-F(x-\epsilon)|=0$$であり,$\delta$は任意だったので,$$\lim_{n\rightarrow\infty}F_n(x)=F(x)$$が成り立つ.従って,$X_n\xrightarrow{d}X$が成り立つ.

\subsection*{4. $X_n\xrightarrow{a.s}X \Rightarrow X_n\xrightarrow{d}X$}
3.より, $X_n\xrightarrow{a.s}X\Rightarrow X_n\xrightarrow{p}X$が成り立ち, さらに$X_n\xrightarrow{p}X\Rightarrow X_n\xrightarrow{d}X$が成り立つので, 連鎖律より, $X_n\xrightarrow{a.s}X$ならば$X_n\xrightarrow{d}X$が成り立つ.
\subsection*{5. $X_n\xrightarrow{L^p}X \Rightarrow X_n\xrightarrow{d}X$}
3.より, $X_n\xrightarrow{L^p}X\Rightarrow X_n\xrightarrow{p}X$が成り立ち, さらに$X_n\xrightarrow{p}X\Rightarrow X_n\xrightarrow{d}X$が成り立つので, 連鎖律より, $X_n\xrightarrow{L^p}X$ならば$X_n\xrightarrow{d}X$が成り立つ.
 
\end{document}