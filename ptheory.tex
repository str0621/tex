\documentclass{jsarticle}
\usepackage[dvipdfmx]{graphicx}
\usepackage[dvipdfmx]{color}
\usepackage{amsmath,amssymb,fancybox,ascmac,booktabs,diagbox,array,otf}
\usepackage{minted}
\setminted{
  linenos,
  breaklines,
  frame=lines,
  framesep=2mm,
}


\author{6324059:塚本 智己}
\title{確率論レポート課題}
\date{提出日:\today}
\begin{document}
\maketitle
%概収束,確率収束,p 次平均収束,分布収束について,定義をまとめよ.
\section{}
\begin{itembox}[l]{問題}
   概収束,確率収束,p次平均収束,分布収束について,定義をまとめよ.
\end{itembox} 

\subsection*{概収束}
(実数値)確率変数列$(X_n)_{n\in \mathbb{N} }$と確率変数列$X$に対して
\begin{equation*}
    P(\lim_{n \rightarrow \infty}X_n = X)=1
\end{equation*}
となるとき, $X_n$は$X$に「概収束(Almost sure Convergence)」するといい, $X_n \xrightarrow{a.s} X (n\rightarrow \infty)$とかく.

\subsection*{確率収束}
$(X_n)_{n\in \mathbb{N} }$が任意の$\epsilon > 0$に対して,
\begin{equation*}
    \lim_{n \rightarrow \infty} P (|X_n-X|>\epsilon)
\end{equation*}
となるとき, $X_n$は$X$に「確率収束(Convergence in Probability)」するといい, 
\begin{gather*}
    P \lim_{n \rightarrow \infty}X_n = X\\
    X_n \xrightarrow{P} X (n \rightarrow \infty)
\end{gather*}
とかく. 
\subsection*{p次平均収束}
$(X_n)_{n\in \mathbb{N} }$, $X$は各$n \in \mathbb{N} $に対してある$p>0$で$E[|X_n|^p]=0$とする. このとき, $$\lim_{n\rightarrow0}E[|X_n-X|^p]=0$$ならば, $X_n$は$X$に「p次平均収束(Convergence in $L^p$)」あるいは$L^p$収束するといい,$$X_n\xrightarrow{L^p}X(n\rightarrow\infty)$$と表す.

\subsection*{分布収束}
$(X_n)_{n\in \mathbb{N} }$, $X$が$\mathbb{R} $上の任意の有界連続関数$f$に対して
$$\lim_{n\rightarrow\infty}E[|f(X_n)|]=E[f(X)]$$
を満たすとき, $X_n$は$X$に「分布収束(Convergence in Distribution)」または「法則収束(Convergence in Law)」するといい,
\begin{gather*} 
X_n \xrightarrow{d}X(n\rightarrow\infty)\\
X_n \Rightarrow X (n\rightarrow \infty)
\end{gather*}
などと表す.

%概収束,確率収束,p 次平均収束,分布収束について,定義をまとめよ.
%上記で述べた,各種収束の関係性をその反例や証明を含めてまとめよ.必要に応じて図などを使用してもよい.
\section{}
\begin{itembox}[l]{問題}
   上記で述べた,各種収束の関係性をその反例や証明を含めてまとめよ.必要に応じて図などを使用してもよい.
\end{itembox}

\begin{center}
\renewcommand{\arraystretch}{1.5} % 行の高さを少し広げる
\begin{tabular}{|c||c|c|c|c|}
    \hline
    % diagbox{左上の見出し}{右下の見出し}
    \diagbox{チーム}{対戦相手} & A & B & C & D \\
    \hline \hline
    A & \diagbox{\phantom{0}}{\phantom{0}} & ○ 3-1 & × 0-2 & ○ 5-0 \\
    \hline
    B & × 1-3 & \diagbox{\phantom{0}}{\phantom{0}} & ○ 2-0 & △ 1-1 \\
    \hline
    C & ○ 2-0 & × 0-2 & \diagbox{\phantom{0}}{\phantom{0}} & × 0-1 \\
    \hline
    D & × 0-5 & △ 1-1 & ○ 1-111 & \diagbox{\phantom{0}}{\phantom{0}} \\
    \hline
\end{tabular}
\end{center}

 
\end{document}